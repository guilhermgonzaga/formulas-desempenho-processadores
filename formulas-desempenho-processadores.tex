% SPDX-License-Identifier: CC0-1.0

\documentclass[12pt]{article}

\usepackage[T1]{fontenc}
\usepackage[utf8]{inputenc}
\usepackage[brazilian]{babel}
\usepackage[colorlinks=false,linktoc=all]{hyperref}
\usepackage[a4paper, margin=25mm]{geometry}
\usepackage{tcolorbox}
\usepackage{parskip}
\usepackage{amsmath}
\usepackage[type={CC},modifier={zero},version={1.0},lang=brazilian]{doclicense}

\tcbset{colback=green!3!white,colframe=green!35!gray,fonttitle=\sc}

\renewcommand{\arraystretch}{1.1}

% Permite adicionar texto itálico com espaços e caracteres UTF-8 no modo matemático
\newcommand{\Text}[1]{\textrm{\textit{#1}}}


\title{Fórmulas para Avaliação de Desempenho de Processadores}
\author{}
\date{\the\month/\the\year}


\begin{document}


\maketitle

\tableofcontents

\vspace*{\fill}

\doclicenseThis

\newpage


\section{Fundamentos}


\subsection{Taxa e período do \textit{clock}}

\[ \Text{taxa do clock} = \frac{1}{\Text{período do clock}} \]


\subsection{Tempo de execução}

Em sistemas modernos, processadores geralmente trabalham em várias tarefas simultaneamente. Isso faz com que o tempo aparente de execução de um programa não necessariamente reflita o seu custo computacional ou o desempenho do processador para aquela tarefa. Portanto, considere duas definições:

\begin{description}
  \item[Tempo real ou decorrido.] O tempo total para executar a tarefa específica, de início a fim.
  \item[Tempo de CPU.] Apenas o tempo em que o processador computa, excluindo tempo de execução de outras tarefas ou espera por recursos, eventos externos e interação do usuário.
\end{description}

O tempo de execução a ser tratado nas próximas seções pode ser um destes, a depender do contexto. Em geral, o tempo de CPU é usado, com exceção de casos em que haja interesse em analisar o desempenho multitarefas do processador.


%\subsection{Desempenho}
%
%\[ \Text{desempenho}_A = \frac{1}{\Text{tempo de execução}_A} \]


\subsection[Prefixos de unidades do SI e IEC]{Prefixos do Sistema Internacional de Unidades (SI) e \textit{International Electrotechnical Commission} (IEC)}

A comunidade científica, por meio do Sistemas Internacional, criou prefixos (constantes numéricas) para facilitar a representação de magnitudes muito grandes ou muito pequenas.

Era comum usar-se os prefixos do SI para potências de 2, dada a proximidade entre \(10^3=1000\) e \(2^{10}=1024\). No entanto, essa diferença cresce exponencialmente conforme as potências aumentam, tornando os prefixos do SI inconvenientes e imprecisos nesse contexto. Por exemplo, existe um erro de apenas 2,4\% entre \(2^{10}\) e \(10^3\), porém o erro sobe para 15,3\% entre \(2^{60}\) e \(10^{18}\).

Por esse motivo, houve um esforço (atual padrão IEC 80000-13) pela \textit{International Electrotechnical Commission} para convencionar constantes úteis no contexto da tecnologia da informação. A tabela~\ref{tab:prefixos} mostra os prefixos mais usados.

\begin{table}[h]
  \centering
  \caption{Prefixos do SI vs. IEC.}\smallskip
  \label{tab:prefixos}
  \begin{tabular}{|ccl|ccc|} \hline
    \multicolumn{3}{|c|}{SI}       & \multicolumn{3}{|c|}{IEC}   \\ \hline
    Nome  & Símbolo & Valor        & Nome & Símbolo & Valor      \\ \hline
%    \multicolumn{6}{c}{\textbf{. . .}} \\ \hline
    yocto & y       & \(10^{-24}\) & ---  & ---     & ---        \\ %\hline
    zepto & z       & \(10^{-21}\) & ---  & ---     & ---        \\ %\hline
    atto  & a       & \(10^{-18}\) & ---  & ---     & ---        \\ %\hline
    femto & f       & \(10^{-15}\) & ---  & ---     & ---        \\ %\hline
    pico  & p       & \(10^{-12}\) & ---  & ---     & ---        \\ %\hline
    nano  & n       & \(10^{-9}\)  & ---  & ---     & ---        \\ %\hline
    micro & \(\mu\) & \(10^{-6}\)  & ---  & ---     & ---        \\ %\hline
    mili  & m       & \(10^{-3}\)  & ---  & ---     & ---        \\ %\hline
    kilo  & k       & \(10^3\)     & kibi & Ki      & \(2^{10}\) \\ %\hline % = 1024^1
    mega  & M       & \(10^6\)     & mebi & Mi      & \(2^{20}\) \\ %\hline % = 1024^2
    giga  & G       & \(10^9\)     & gibi & Gi      & \(2^{30}\) \\ %\hline % = 1024^3
    tera  & T       & \(10^{12}\)  & tebi & Ti      & \(2^{40}\) \\ %\hline % = 1024^4
    peta  & P       & \(10^{15}\)  & pebi & Pi      & \(2^{50}\) \\ %\hline % = 1024^5
    exa   & E       & \(10^{18}\)  & exbi & Ei      & \(2^{60}\) \\ %\hline % = 1024^6
    zetta & Z       & \(10^{21}\)  & zebi & Zi      & \(2^{70}\) \\ %\hline % = 1024^7
    yotta & Y       & \(10^{24}\)  & yobi & Yi      & \(2^{80}\) \\ \hline % = 1024^8
%    \multicolumn{6}{c}{\textbf{. . .}} \\
  \end{tabular}
\end{table}

Há situações em que é preferível usar um conjunto de prefixos ao outro, devido à natureza dos valores tratados. Por exemplo:

\begin{minipage}[t]{0.5\linewidth}
SI
\begin{itemize}
  \item Grandezas das ciências da natureza
  \item Largura de banda
  \item Taxa de transferência
  \item Tempo
\end{itemize}
\end{minipage}%
\begin{minipage}[t]{0.5\linewidth}
IEC
\begin{itemize}
  \item Capacidade de armazenamento
\end{itemize}
\end{minipage}

\begin{tcolorbox}[title=Dicas,colback=black!3!white,colframe=black!25!white]
  Caso esteja curioso: a letra \emph{i} em \emph{Ki}, \emph{Mi} etc. vem de \emph{b\textbf{i}nary}.
  \tcblower
  Quer um conselho? Não precisa pronunciar os nomes como a IEC definiu, apenas use os símbolos corretos.
\end{tcolorbox}


\section{\textit{Speedup}}


\begin{tcolorbox}[title=Lembre-se]
  \textit{speedup} é uma medida que varia de acordo com o código executado.
\end{tcolorbox}

\begin{equation} \label{eq:speedup}
  \Text{speedup}_{A/B} =
%  \frac{\Text{desempenho}_A}{\Text{desempenho}_B} =
  \frac{\Text{tempo de execução}_B}{\Text{tempo de execução}_A}
\end{equation}

Lê-se \emph{\textit{speedup de A com relação a B}}. Alternativamente:

\[
  \Text{tempo de execução}_A
  = \frac{\Text{tempo de execução}_B}{\Text{speedup}_{A/B}}
\]

%\[ \Text{desempenho}_A = \Text{speedup}_{A/B} \times \Text{desempenho}_B \]

Dada a natureza do cálculo do \textit{speedup}, existe também a seguinte propriedade.

\[ \Text{speedup}_{A/B} = \frac{1}{\Text{speedup}_{B/A}} \]


\subsection{\textit{Speedup} calculado com modelo de referência}

Considere dois processadores, \emph{A} e \emph{B}, e um programa \emph{P}. Deseja-se saber o valor de \( \Text{speedup}_{A/B} \) para \emph{P}, mas não se sabe os tempos de execução do programa em ambos os processadores.

No entanto, existe um processador, \emph{X}, tal que \( \Text{speedup}_{A/X} \) e \( \Text{speedup}_{B/X} \) para \emph{P} são conhecidos. Nesta situação, pode-se aproveitar a seguinte propriedade.

\begin{align*}
  \Text{speedup}_{A/B}
  &= \frac{\Text{tempo de execução}_B}{\Text{tempo de execução}_A} \\[\medskipamount]
  &= \frac{\Text{tempo de execução}_X}{\Text{speedup}_{B/X}} \times
     \frac{\Text{speedup}_{A/X}}{\Text{tempo de execução}_X}\\[\medskipamount]
  &= \frac{\Text{speedup}_{A/X}}{\Text{speedup}_{B/X}}
\end{align*}

%\begin{align*}
%  \Text{speedup}_{A/B}
%  &= \frac{\Text{desempenho}_A}{\Text{desempenho}_B} \\[\medskipamount]
%  &= \frac{
%      \Text{speedup}_{A/X} \times \Text{desempenho}_X
%    }{
%      \Text{speedup}_{B/X} \times \Text{desempenho}_X
%    } \\[\medskipamount]
%  &= \frac{\Text{speedup}_{A/X}}{\Text{speedup}_{B/X}}
%\end{align*}

Isso significa que o \textit{speedup} de um processador com relação a outro pode ser obtido com apenas os respectivos \textit{speedups} destes com relação a um terceiro processador qualquer.


\section{Ciclos por Instrução (CPI)}


\begin{tcolorbox}[title=Lembre-se]
  O CPI varia de acordo com o tipo de instrução executada.
\end{tcolorbox}

\begin{equation} \label{eq:numciclosclock}
  \Text{nº de ciclos gastos} = \Text{nº de instruções executadas} \times \Text{CPI}
\end{equation}


\subsection{Cálculo do CPI médio}

\begin{tcolorbox}[title=Lembre-se]
  O CPI médio varia de acordo com o código executado.
\end{tcolorbox}

Considere:
\begin{itemize}
  \item \( C_i \): nº de instruções do tipo \emph{i} executadas
  \item \( CPI_i \): nº de ciclos gastos com uma instrução do tipo \emph{i}
  \item \( \Text{nº de ciclos gastos} = \sum_{\forall i} (C_i \times CPI_i) \)
  \item \( \Text{nº de instruções executadas} = \sum_{\forall i} C_i \)
\end{itemize}

\[
  CPI = \frac{\Text{nº de ciclos gastos}}{\Text{nº de instruções executadas}}
  = \frac{\sum_{\forall i} (C_i \times CPI_i)}{\sum_{\forall i} C_i}
\]

Note que a equação acima é apenas a média ponderada dos CPI de cada tipo de instrução, onde os pesos são as proporções (\%) de cada tipo de instrução no código executado.


\subsection{Instruções por ciclo (IPC)}

Além do CPI, pode-se também analisar o desempenho considerando a razão IPC. A escolha é meramente arbitrária, e por vezes uma alternativa pode ser mais intuitiva que a outra.

As equações mudam um pouco com IPC, mas basta lembrar que este é o inverso do CPI.

\[ IPC = \frac{1}{CPI} \]


\section{Equação de Desempenho do Processador}

\begin{tcolorbox}[title=Lembre-se]
  Um processador pode executar vários trechos de código ao mesmo tempo, sejam do seu programa ou não. Portanto, o tempo de CPU não inclui a execução de outros programas nem a espera por dispositivos de entrada e saída.
\end{tcolorbox}

\begin{align} \label{eq:tempocpu}
  \Text{tempo}_{CPU} &= \Text{nº de ciclos gastos} \times \Text{período do clock} \\[\medskipamount]
  &= \frac{\Text{nº de ciclos gastos}}{\Text{taxa do clock}} \nonumber
\end{align}

Substituindo a equação \ref{eq:numciclosclock} em \ref{eq:tempocpu}, obtemos

\begin{align} \label{eq:desempenho}
  \Text{tempo}_{CPU}
  &= \Text{nº de instruções executadas} \times \Text{CPI} \times \Text{período do clock} \\[\medskipamount]
  &= \frac{\Text{nº de instruções executadas} \times \Text{CPI}}{\Text{taxa do clock}} \nonumber
\end{align}


\section{Lei de Amdahl}


Considere:
\begin{itemize}
  \item \(f \in [0, 1]\): fração do tempo de execução original afetada pela melhoria
  \item \(p\): proporção da melhoria em comparação com o mesmo trecho da execução original
\end{itemize}

\begin{align} \label{eq:leiamdahl}
  \Text{tempo com melhoria}
  &= \Text{tempo sem melhoria}_\Text{não afetado}
  + \frac{\Text{tempo sem melhoria}_\Text{afetado}}{\Text{proporção da melhoria}} \\[\medskipamount]\nonumber
  &= \Text{tempo sem melhoria} \times (1 - f)
  + \frac{\Text{tempo sem melhoria} \times f}{p}
\end{align}

Manipulando-se a equação \ref{eq:leiamdahl} com o tempo de execução original, encontramos uma nova equação.

\begin{equation} \label{eq:speedupamdahl}
  speedup = \frac{1}{(1 - f) + \frac{f}{p}}
\end{equation}


\end{document}
