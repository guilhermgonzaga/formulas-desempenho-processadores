\documentclass[12pt]{article}

\usepackage[T1]{fontenc}
\usepackage[utf8]{inputenc}
\usepackage[brazilian]{babel}
\usepackage[colorlinks=false,linktoc=all]{hyperref}
\usepackage[a4paper, margin=2.6cm]{geometry}
\usepackage{parskip}
\usepackage{amsmath}
\usepackage[type={CC},modifier={zero},version={1.0},lang=brazilian]{doclicense}

\renewcommand{\arraystretch}{1.1}

% Permite adicionar texto itálico com espaços e caracteres UTF-8 no modo matemático
\newcommand{\Text}[1]{\textrm{\textit{#1}}}


% TODO Adicionar fórmulas de stall e desempenho da hierarquia de memórias


\title{Fórmulas para Avaliação de Desempenho de Processadores}
\author{}
\date{}


\begin{document}


\maketitle

\tableofcontents

\vspace*{\fill}

\doclicenseThis

\newpage


\section{Fundamentos}


\subsection{Taxa e período do \textit{clock}}

\[ \Text{taxa do clock} = \frac{1}{\Text{período do clock}} \]


\subsection{Desempenho e tempo de execução}

\[ \Text{desempenho}_A = \frac{1}{\Text{tempo de execução}_A} \]


\subsection[Prefixos do SI e IEC]{Prefixos do Sistema Internacional de Unidades (SI) e \textit{International Electrotechnical Commission} (IEC)}

A comunidade científica, por meio do Sistemas Internacional, criou prefixos (constantes numéricas) para facilitar a representação de magnitudes muito grandes ou muito pequenas.

Igualmente, houve pela \textit{International Electrotechnical Commission} um esforço para criar constantes úteis no contexto da computação, pois a forma como circuitos digitais são implementados (álgebra booleana, aritmética em base binária etc.) torna os prefixos do SI por vezes inconvenientes.

Pela proximidade entre $10^3=1000$ e $2^{10}=1024$, definiu-se prefixos a partir de $2^{10}$ para manter certa similaridade entre os dois conjuntos. A tabela~\ref{tab:prefixos} mostra os prefixos mais usados, bem como situações em que é preferível usar um conjunto de prefixos ao outro.

\begin{table}[h]
  \centering
  \caption{Prefixos do SI e IEC}\smallskip
  \label{tab:prefixos}
  \begin{tabular}{|c|c|l|c|c|} \hline
    & \multicolumn{2}{|p{3.4cm}|}{Para tempo, velocidade ou taxa de transferência (SI)} & \multicolumn{2}{|p{2.8cm}|}{Para tamanho ou capacidade (IEC)} \\ \hline
    Prefixo & Símbolo & Valor    & Símbolo & Valor          \\ \hline
    \multicolumn{5}{c}{\textbf{. . .}} \\ \hline
    pico    & p       & \(10^{-12}\) & ---     & ---                 \\ %\hline
    nano    & n       & \(10^{-9}\)  & ---     & ---                 \\ %\hline
    micro   & \(\mu\) & \(10^{-6}\)  & ---     & ---                 \\ %\hline
    mili    & m       & \(10^{-3}\)  & ---     & ---                 \\ %\hline
    kilo    & K ou k  & \(10^3\)     & Ki      & \(2^{10}\) \\ %\hline % = 1024^1
    Mega    & M       & \(10^6\)     & Mi      & \(2^{20}\) \\ %\hline % = 1024^2
    Giga    & G       & \(10^9\)     & Gi      & \(2^{30}\) \\ %\hline % = 1024^3
    Tera    & T       & \(10^{12}\)  & Ti      & \(2^{40}\) \\ %\hline % = 1024^4
    Peta    & P       & \(10^{15}\)  & Pi      & \(2^{50}\) \\ %\hline % = 1024^5
    Exa     & E       & \(10^{18}\)  & Ei      & \(2^{60}\) \\ \hline % = 1024^6
    \multicolumn{5}{c}{\textbf{. . .}} \\
  \end{tabular}
\end{table}


\section{\textit{Speedup}}


Lembre-se: \textit{speedup} é uma medida que varia de acordo com o programa executado.

\begin{equation} \label{eq:speedup}
  \Text{speedup}_{A/B} =
  \frac{\Text{desempenho}_A}{\Text{desempenho}_B} =
  \frac{\Text{tempo de execução}_B}{\Text{tempo de execução}_A}
\end{equation}

Alternativamente:

\[ \Text{desempenho}_A = \Text{speedup}_{A/B} \times \Text{desempenho}_B \]

\[
  \Text{tempo de execução}_A
  = \frac{\Text{tempo de execução}_B}{\Text{speedup}_{A/B}}
\]

Dada a natureza do cálculo do \textit{speedup}, existe também a seguinte propriedade.

\[ \Text{speedup}_{A/B} = \frac{1}{\Text{speedup}_{B/A}} \]


\subsection{\textit{Speedup} calculado com modelo de referência}

Considere dois processadores, \textit{A} e \textit{B}, e um programa \textit{P}. Deseja-se saber o valor de \( \Text{speedup}_{A/B} \) para \textit{P}, mas não se sabe os tempos de execução do programa em ambos os processadores.

No entanto, existe um processador, \textit{X}, tal que \( \Text{speedup}_{A/X} \) e \( \Text{speedup}_{B/X} \) para \textit{P} são conhecidos. Nesta situação, pode-se aproveitar a seguinte propriedade.

\begin{align*}
  \Text{speedup}_{A/B}
  &= \frac{\Text{desempenho}_A}{\Text{desempenho}_B} \\[\medskipamount]
  &= \frac{
      \Text{speedup}_{A/X} \times \Text{desempenho}_X
    }{
      \Text{speedup}_{B/X} \times \Text{desempenho}_X
    } \\[\medskipamount]
  &= \frac{\Text{speedup}_{A/X}}{\Text{speedup}_{B/X}}
\end{align*}

%\[ % Forma alternativa
%  \Text{speedup}_{A/B}
%  = \frac{\Text{desempenho}_A}{\Text{desempenho}_B}
%  = \frac{
%      \left(\Text{speedup}_{A/X} \times \Text{desempenho}_X\right)
%    }{
%      \left(\Text{speedup}_{B/X} \times \Text{desempenho}_X\right)
%    }
%  = \frac{\Text{speedup}_{A/X}}{\Text{speedup}_{B/X}}
%\]

Isso significa que o \textit{speedup} de um processador com relação a outro pode ser obtido com apenas os respectivos \textit{speedups} destes com relação a um terceiro processador qualquer.


\section{Ciclos por Instrução (CPI)}


Lembre-se: o CPI varia de acordo com o tipo de instrução executada.

\begin{equation} \label{eq:numciclosclock}
  \Text{nº de ciclos gastos} = \Text{nº de instruções executadas} \times \Text{CPI}
\end{equation}


\subsection{Cálculo do CPI Médio}

Lembre-se: o CPI médio varia de acordo com o código executado. Considere:
\begin{itemize}
  \item \( C_i \): nº de instruções do tipo \textit{i} executadas
  \item \( CPI_i \): nº de ciclos gastos com uma instrução do tipo \textit{i}
  \item \( \Text{nº de ciclos gastos} = \sum_{\forall i} (C_i \times CPI_i) \)
  \item \( \Text{nº de instruções executadas} = \sum_{\forall i} C_i \)
\end{itemize}

\[
  CPI = \frac{\Text{nº de ciclos gastos}}{\Text{nº de instruções executadas}}
  = \frac{\sum_{\forall i} (C_i \times CPI_i)}{\sum_{\forall i} C_i}
\]

Note que a equação acima é apenas a média ponderada dos CPI de cada tipo de instrução, onde os pesos são as proporções (\%) de cada tipo de instrução no código executado.


\section{Equação de Desempenho do Processador}


Lembre-se: o processador executa vários trechos de código ao mesmo tempo, sejam do seu programa ou não. Portanto, o tempo de CPU não inclui a execução de outros programas nem a espera por dispositivos de entrada e saída.

\begin{align} \label{eq:tempocpu}
  \Text{tempo}_{CPU} &= \Text{nº de ciclos gastos} \times \Text{período do clock} \\[\medskipamount]
  &= \frac{\Text{nº de ciclos gastos}}{\Text{taxa do clock}} \nonumber
\end{align}

Substituindo a equação \ref{eq:numciclosclock} em \ref{eq:tempocpu}, obtemos

\begin{align} \label{eq:desempenho}
  \Text{tempo}_{CPU}
  &= \Text{nº de instruções executadas} \times \Text{CPI} \times \Text{período do clock} \\[\medskipamount]
  &= \frac{\Text{nº de instruções executadas} \times \Text{CPI}}{\Text{taxa do clock}} \nonumber
\end{align}


\section{Lei de Amdahl}


Considere:
\begin{itemize}
  \item \(f \in [0, 1] \): fração do tempo de execução do programa a ser melhorada
  \item \(p\): proporção da melhoria, se comparado ao programa sem a melhoria
\end{itemize}

\begin{align} \label{eq:leiamdahl}
  \Text{tempo com melhoria}
  &= \Text{tempo sem melhoria}_\Text{não afetado}
  + \frac{\Text{tempo sem melhoria}_\Text{afetado}}{\Text{proporção da melhoria}} \\[\medskipamount]\nonumber
  &= \Text{tempo sem melhoria} \times (1 - f)
  + \frac{\Text{tempo sem melhoria} \times f}{p}
\end{align}

Manipulando-se a equação \ref{eq:leiamdahl} com o tempo de execução do programa sem a melhoria, encontra-se uma nova equação.

\begin{equation} \label{eq:speedupamdahl}
  speedup = \frac{1}{(1 - f) + \frac{f}{p}}
\end{equation}


\end{document}
